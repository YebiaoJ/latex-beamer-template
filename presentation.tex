\documentclass[xcolor=table]{beamer}

% This is a University of Pennsylvania branded LaTeX Beamer template that can be used for creating
% a PDF presentation slide deck
% 
% Credit for this Beamer theme and TeX document:
%
% Benjamin French, PhD  
% Clay Wells


% Add/remove packages you need/don't need
\usepackage{setspace, graphics, hyperref, bm, tabularx, enumerate, booktabs, caption, verbatim}

% UPenn specific theme
\usetheme{Penn}

% General settings
\setbeamercovered{transparent}
\setbeamertemplate{enumerate item}[default]
\setbeamertemplate{section in toc}[default]
\setbeamertemplate{subsection in toc}[default]
\setbeamertemplate{navigation symbols}{}

\AtBeginSection[] 
{ 
\begin{frame}<beamer>{Overview} 
\tableofcontents[currentsection] 
\end{frame} 
} 

%% UPDATE with your details. I use a different author section.. experiment
%% Set presentation information
\title[Beamer Themes]{\textbf{Creating a Beamer Template}}

\author[F Last]{\textbf{First Last} \\ 
	{\small Department of Something or Other} \\
	{\small Your Organization} \\
	 \textbf{username@organization.edu }}

\institute[username@organization.edu]{}

\date[1 January 1999]{Security-SIG \\ 1 January 1999 }


%% Begin the presentation document
%% This is for demonstration purposes only. Take what you like and remove the rest.
%% You may notice there are a few ways of doing the same thing, e.g. frametitles

\begin{document}

\frame[plain]{\titlepage}

% Don't include the titlepage in the total slide count.
\setcounter{framenumber}{0}

% Table of Contents slide
\begin{frame}
\tableofcontents
\end{frame}

\begin{frame}
\frametitle{Disclaimers}
\begin{enumerate}
\item<1-> Title of this seminar should have been Creating a Beamer Theme
\medskip
\item<2-> I assume you know something about \LaTeX~and Beamer
\medskip
\item<3-> This will all appear to be a bit hacky\ldots
\end{enumerate}
\end{frame}

\section{Themes}
\begin{frame}
\frametitle{Themes}
\begin{itemize}
\item The Penn theme I ``created'' is a \textcolor{quakerblue}{global} or \textcolor{quakerblue}{presentation} theme 
\smallskip
\item Mostly, it specifies the \textcolor{quakerblue}{outer} and \textcolor{quakerblue}{inner} themes, calls out the \textcolor{quakerblue}{color} theme, and structures the title page \\
\begin{itemize}
\item \textcolor{quakerblue}{Outer}: Dictates the style of head and foot lines, sidebars
\item \textcolor{quakerblue}{Inner}: Dictates the style of elements inside the frame, \\ such as environments (itemize, enumerate, theorem)
\item \textcolor{quakerblue}{Color}: Dictates the color scheme for all elements
\item \textcolor{quakerblue}{Font}: Use the default (never use serif fonts in presentations)
\end{itemize}
\smallskip
\item These options are contained in \texttt{.sty} files
\smallskip
\item \href{http://www.upenn.edu/webservices/styleguide/logo.html}{\alert{Penn logos are available for download}}
\end{itemize}
\end{frame}


\definecolor{dodgerblue3}{RGB}{24,116,205}

\begin{frame}[fragile]
\frametitle{Color: Going both ways}
First in R\ldots
\begin{small}
\begin{verbatim}
colors()[131]
[1] "dodgerblue3"

col2rgb("dodgerblue3")
      [,1]
red     24
green  116
blue   205
\end{verbatim}
\end{small}
\ldots then in Beamer
\begin{small}
\begin{verbatim}
\definecolor{dodgerblue3}{RGB}{24, 116, 205}
\textcolor{dodgerblue3}{This is dodgerblue3!}
\end{verbatim}
\end{small}
\textcolor{dodgerblue3}{This is dodgerblue3!}
\end{frame}

\section{Beamer Resources}
\begin{frame}
\frametitle{Resources}
\begin{itemize}
\item \href{http://www.cpt.univ-mrs.fr/~masson/latex/Beamer-appearance-cheat-sheet.pdf}{\alert{Beamer appearance cheat sheet}}
\item \href{http://texdoc.net/texmf-dist/doc/latex/beamer/doc/beameruserguide.pdf}{\alert{User's guide to the Beamer class}}
\end{itemize}
\end{frame}

%%%%
%% Adding a few slides to demonstrate a slightly different approach.

%% A Brief History
\section{A Brief History} %% sections are required for the purpose of a TOC
\begin{frame}{A Brief History}
A Linux distribution with a focus on penetration testing tasks.
\medskip
\begin{itemize}
\item Previously known as BackTrack
\item BackTrack - Whoppix, IWHAX, and Auditor
\item Kali 1.0 released March 12, 2013
\item Kali 2.0 release August 11, 2015
\end{itemize}
\end{frame}

\section{Features}

%% Features
\begin{frame}{Features}
\medskip
\begin{itemize}
\item Based on Debian Linux distribution
\item 100's of PenTesting tools
\item Vast wireless card support
\item Custom kernel patched for packet injection
\item All packages GPG signed by developers
\item Highly customizable 
\item Supports ARM-based systems
\end{itemize}
\end{frame}

%%%%%

\newcounter{finalframe}
\setcounter{finalframe}{\value{framenumber}}

\begin{frame}[fragile]
\frametitle{Frame numbers}
\begin{itemize}
\item Do not count the title page (or table of contents) 
\begin{verbatim}
\frame[plain]{\titlepage}
\setcounter{framenumber}{0}
\end{verbatim}
\smallskip
\item Set the total number of frames prior to any appendices
\begin{verbatim}
\newcounter{finalframe}
\setcounter{finalframe}{\value{framenumber}}
% Supplementary frames
\setcounter{framenumber}{\value{finalframe}}
\end{verbatim}
\end{itemize}
\end{frame}

\setcounter{framenumber}{\value{finalframe}}

\end{document}
